\documentclass[12pt]{article}
\usepackage[utf8]{inputenc}
\usepackage{float}
\usepackage{amsmath}


\usepackage[hmargin=3cm,vmargin=6.0cm]{geometry}
%\topmargin=0cm
\topmargin=-2cm
\addtolength{\textheight}{6.5cm}
\addtolength{\textwidth}{2.0cm}
%\setlength{\leftmargin}{-5cm}
\setlength{\oddsidemargin}{0.0cm}
\setlength{\evensidemargin}{0.0cm}




\begin{document}


normal text \\ new line

\textbf{bold}, \textit{italic} 

\textbf{\textit{bold $\wedge$ italic }}

special characters usage : \% \&

text character x vs. variable $x$

\textbf{sub} : $x_i$, $s'_i$ , \textbf{hyper} : $x^k$, \textbf{both} $x_i^k$

\vspace*{20px} %use for spaces between, user defined number



big operator with lower and upper indexes : $\sum_{i=0}^{K} x_i^t$

you cannot put spaces in math mode $x   =    y   +   4$

you can put spaces with special characters using :
\begin{verbatim}
 \ \quad \qquad 
\end{verbatim}

example of spacing in math mode $ x \ = \quad x \qquad + \ 1$

equation and referencing the equation :


\begin{equation} 
\label{eu_eqn}
e^{\pi i} - 1 = 0
\end{equation}
 
The beautiful equation \ref{eu_eqn} is known as the Euler equation

aligned multiple lines :

\begin{equation}
\label{equation:value2}
\begin{split}
V^{\pi}(s) &= E_{\pi}[r_{t+1} + \gamma r_{t+2} + \gamma^{2}r_{t+3} ... | s_t = s] ,
\\         &= E_{\pi}[r_{t+1} | s_t = s ] + \gamma E_{\pi}[r_{t+2} + \gamma r_{t+3} ... | s_t = s] ,
\\         &= E_{\pi}[r_{t+1} + \gamma V^{\pi}(s_{t+1}) | s_t] , 
\end{split}
\end{equation}

\begin{equation} 
\label{eq1}
\begin{split}
A & = \frac{\pi r^2}{2} \\
 & = \frac{1}{2} \pi r^2
\end{split}
\end{equation}

aligning several equations :

\begin{align*}
x&=y           &  w &=z              &  a&=b+c\\
2x&=-y         &  3w&=\frac{1}{2}z   &  a&=b\\
-4 + 5x&=2+y   &  w+2&=-1+w          &  ab&=cb
\end{align*}

grouping and centering equations :

\begin{gather*} 
2x - 5y =  8 \\ 
3x^2 + 9y =  3a + c
\end{gather*}

\vspace*{30px}

itemize !!! student's best friend for increasing page count

\begin{itemize}
\item item 1
\item item 2 

\end{itemize}


tables, easy construction yet exhausting due to placement issues, even though this text 
is written above the table \ref{table:example} in the code, latex places the table 
on top of the page.

\begin{table}
\small
\centering
\caption{ example table for $foo$ }
\label{table:example}
\begin{tabular}{|c|c|c|c|c|c|}	%% specify column number
\hline 							%% line draw
\textbf{title1} & \textbf{title2} & \textbf{title3} & \textbf{title4} & \textbf{title5} & \textbf{title6} \\
\hline 
\hline 
text & 0 & text & 20 & text & $a_0$\\			%% rows distinguished with &
text & 0 & & & & \\
text & 0 & $foo$ & 0 & $\pi(s = 1)$ & $a_2$\\
 &  & $foo$ & 0 & $\pi(s = 3)$ & $\emptyset$\\


\hline 
\end{tabular}
\end{table}


if you want to put table \ref{table:put-here} here put the option [H] in the $begin$ tag in the code :

\begin{table}[H]
\small
\centering
\caption{beautiful caption }
\label{table:put-here}
\begin{tabular}{|c|c|c|c|c|c|c|c|} %% specify column number
\cline{2-7} 						%% specify column line
\multicolumn{1}{c|}{} & \multicolumn{6}{c|}{\textbf{Multi-Column Title}} & \multicolumn{1}{c}{}\\
\cline{2-8} 						
\multicolumn{1}{c|}{} & title & title & title & title & title & title & title\\
\hline 
$foo$ &  & text & text & text &  &  & \\
$hoo$ & text & text & text &  & text &  & \\
\hline
\textbf{$foohoo$} & text & text & text & text & text &  & \textbf{83\%}\\
\hline 
\end{tabular}
\end{table}


$\hat{\sigma}_{\hat{m}_mle}^2 = \frac{\hat{\theta}_mle^2}{n} [1 + \frac{6(1 - \epsilon)^2}{\pi^2}] = \frac{1.10867 \hat{\theta}_mle^2}{n} \hat{\sigma}_{\hat{\theta}_mle}^2 = \frac{6}{\pi^2} \frac{\hat{\theta}_mle^2}{n} = \frac{0.60793 \hat{\theta}_mle^2}{n}
$


\end{document}

​



